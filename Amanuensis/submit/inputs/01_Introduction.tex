%%% %%%%%%%%%%%%%%%%%%%%%%%%%%%%%%%%%%%%%%%%%%%%%%%%%%%%%%%%%%%%%%%%%%%%%%%%%%%%

\section{Introduction: Programmer's Apprentice}

%%% %%%%%%%%%%%%%%%%%%%%%%%%%%%%%%%%%%%%%%%%%%%%%%%%%%%%%%%%%%%%%%%%%%%%%%%%%%%%

Suppose you could merely imagine a computation, and a digital prostheses, an extension of your biological brain, would turn it into code that instantly realizes what you had in mind. Imagine looking at an image, dataset or set of equations and wanting to analyze and explore its meaning as an artistic whim or part of a scientific investigation. I don't mean you would use an existing software suite to produce a standard visualization, but rather you would make use of an extensive repository of existing code to assemble a new program analogous to how a composer draws upon a repertoire of musical motifs, themes and styles to construct new works, and tantamount to having a talented musical amanuensis who, in addition to copying your scores, takes liberties with your prior work, making small alterations here and there and occasionally adding new works of its own invention, novel but consistent with your taste and sensibilities.

Perhaps the interaction would be wordless and you would express your objective by simply focusing your attention and guiding your imagination, the prostheses operating directly on patterns of activation arising in your primary sensory, proprioceptive and associative cortex that have become part of an extensive vocabulary that you now share with your personal digital amanuensis. Or perhaps it would involve a conversation conducted in subvocal, unarticulated speech in which you specify what it is you want to compute and your assistant asks questions to clarify your intention and the two of you share examples of input and output to ground your internal conversation in concrete terms. 

More than thirty years ago, Charles Rich and Richard Waters published an MIT AI Lab technical report~\cite{RichandWatersAIM-87} entitled {\it{The Programmer's Apprentice: A Research Overview}}. Whether they intended it or not, it would have been easy in those days for someone to misremember the title and inadvertently refer to it as "The Sorcerer's Apprentice" since computer programmers at the time were often characterized as wizards and most children were familiar with the Walt Disney movie {\it{Fantasia}}, featuring music written by Paul Dukas inspired by Goethe's poem of the same name. The Rich and Waters conception of an apprentice was certainly more prosaic than the idea described above, but they might have had trouble anticipating the amount of code available in open-source repositories and the considerable computational power we carry about on our persons or can access through the cloud.

In any case, you might find it easier to imagine describing programs in natural language and supplementing your descriptions with input-output pairs. The programs could be as simple as regular expressions or SQL queries or as complicated as designing powerful simulators and visualization algorithms. The point is that there is a set of use cases that are within our reach now and that set will grow as we improve our natural language understanding and machine learning tools. I simply maintain that the scope of applications within reach today is probably larger than you think and that our growing understanding of human cognition is helping to substantially broaden that scope and significantly improve the means by which we interact with computers in general and a new generation of digital prostheses in particular. Here are just a few of the implications that might follow from pursuing a very practical and actionable modern version of The Programmer's Apprentice:

%%% %%%%%%%%%%%%%%%%%%%%%%%%%%%%%%%%%%%%%%%%%%%%%%%%%%%%%%%%%%%%%%%%%%%%%%%%%%%%

{\bf{Develop systems that enable human-machine collaboration on challenging design problems including software engineering:}} The objective of this effort is to develop digital assistants that learn from continuous dialog with an expert software engineer while providing initial value as powerful analytical, computational and mathematical savants. Over time these savants learn cognitive strategies (domain-relevant problem solving skills) and develop intuitions (heuristics and the experience necessary for applying them) by learning from their expert associates. By doing so these savants elevate their innate analytical skills allowing them to partner on an equal footing as versatile collaborators \emdash{} effectively serving as cognitive extensions and digital prostheses, thereby amplifying and emulating their human partner's conceptually-flexible thinking patterns and enabling improved access to and control over powerful computing resources. 

%%% %%%%%%%%%%%%%%%%%%%%%%%%%%%%%%%%%%%%%%%%%%%%%%%%%%%%%%%%%%%%%%%%%%%%%%%%%%%%

{\bf{Leverage and extend the current state of the art in machine learning by integrating human and machine intelligence:}} Current methods for training neural networks typically require substantial amounts of carefully labeled and curated data. Moreover the environments in which many learning systems are expected to perform are partially observable and non-stationary. The distributions that govern the presentation of examples change over time requiring constant effort to collect new data and retrain. The ability to solicit and incorporate knowledge gleaned from new experience to modify subsequent expectations and adapt behavior is particularly important for systems such as digital assistants with whom we interact and routinely share experience. Effective planning and decision making rely on counterfactual reasoning in which we imagine future states in which propositions not currently true are accommodated or steps taken to make them true~\cite{HassabisandMaguireTiCS-07}. The ability for digital assistants to construct predictive models of other agents \emdash{} so-called theory-of-mind modeling \emdash{} is critically important for collaboration~\cite{RabinowitzetalCoRR-18}.

%%% %%%%%%%%%%%%%%%%%%%%%%%%%%%%%%%%%%%%%%%%%%%%%%%%%%%%%%%%%%%%%%%%%%%%%%%%%%%%

{\bf{Draw insight from cognitive and systems neuroscience to implement hybrid connectionist and symbolic reasoning systems:}} Many state-of-the-art machine learning systems now combine differentiable and non-differentiable computational models. The former consist of fully-differentiable connectionist artificial neural networks. They achieve their competence by leveraging a combination of distributed representations facilitating context-sensitive, noise-tolerant pattern-recognition and end-to-end training via backpropagation. The latter, non-differentiable models, excel at manipulating representations that exhibit combinatorial syntax and semantics, are said to be full systematic and compositional, and can directly and efficiently exploit the advantages of traditional von Neumann computing architectures. The differences between the two models are at the heart of the connectionist versus symbolic systems debate that dominated cognitive science in 80's and continues to this day~\cite{OReillyetalTACO-14,FodorandPylyshynCOGNITION-88}. Rather than simulate symbolic reasoning within connectionist models or vice a versa, we simply acknowledge their strengths and build systems that enable efficient integration of both types of reasoning.

%%% %%%%%%%%%%%%%%%%%%%%%%%%%%%%%%%%%%%%%%%%%%%%%%%%%%%%%%%%%%%%%%%%%%%%%%%%%%%%

{\bf{Take advantage of advances in natural language processing to implement systems capable of continuous focused dialog:}} Language is arguably the most important technical innovation in the history of humanity. Not only does it make possible our advanced social skills, but it allows us to pass knowledge from one generation to the next and provides the foundation for mathematical and logical reasoning. Natural language is our native programming language. It is the way we communicate plans and coordinate their execution. In terms of expressiveness, it surpasses modern computer programming languages, but its capability for communicating imprecisely and even incoherently, and our tendency for utilizing that capability makes it a poor tool for programming conventional computers. That said it serves us well in training scientists and engineers to develop and apply more precise languages, and its expressiveness along with our facility using it make it an ideal means for humans and AI systems to collaborate. The consolidation and subsequent recall and management of episodic memory is a key part of what makes us human and enables our diverse social behaviors. Episodic memory makes it possible to create and maintain long-term relationships and collaborations~\cite{PritzeletalICML-17,MoscovitchetalARP-16,OReillyetalCS-15}.

%%% %%%%%%%%%%%%%%%%%%%%%%%%%%%%%%%%%%%%%%%%%%%%%%%%%%%%%%%%%%%%%%%%%%%%%%%%%%%%

{\bf{Think seriously about how such technology might ultimately be employed to build brain-computer-interfaced prostheses:}} This exercise primarily relies on the use of natural language to facilitate communication between the expert programmer and apprentice AI system. The AI system learns to use natural language in much the same way as a human apprentice would \emdash{} as a flexible and expressive tool to capture and convey understanding and recognize and resolve misunderstanding and ambiguity. The AI system interacts with computing hardware through a highly instrumented integrated development environment. Essentially, the AI system can read, write, execute and debug code by simply thinking \emdash{} reading and writing to a differentiable neural computing interface~\cite{GravesetalNATURE-16}. It can also directly sense code running by reading from {\tt{STDERR}} and {\tt{STDIO}}, parsing output from the debugger and collecting and analyzing program traces. The same principles could be applied to develop digital prostheses employed for a wide range of intelligence-enhancing human-computer interfaces.

%%% %%%%%%%%%%%%%%%%%%%%%%%%%%%%%%%%%%%%%%%%%%%%%%%%%%%%%%%%%%%%%%%%%%%%%%%%%%%%

\subsection{Resources}

%%% %%%%%%%%%%%%%%%%%%%%%%%%%%%%%%%%%%%%%%%%%%%%%%%%%%%%%%%%%%%%%%%%%%%%%%%%%%%% 

This document attempts to optimize for the student or software engineer knowledgeable about neural networks and interested primarily in understanding how one might go about building a system along the lines of the programmer's apprentice. It is my experience that this audience has relatively little appetite for details about relevant work in cognitive and systems neuroscience that has informed the design sketched in these pages. The course website for the class I taught at Stanford in the 2018 Spring quarter serves as an extensive resource for those reading this document. It includes all of the {\urlh{https://web.stanford.edu/class/cs379c/calendar.html}{lectures}} and {\urlh{https://web.stanford.edu/class/cs379c/class_messages_listing/index.html}{discussion notes}} for the class and I refer to it often in this document as a source of supplementary information.

%%% %%%%%%%%%%%%%%%%%%%%%%%%%%%%%%%%%%%%%%%%%%%%%%%%%%%%%%%%%%%%%%%%%%%%%%%%%%%% 
