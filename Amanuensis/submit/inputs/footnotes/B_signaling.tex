%%% %%%%%%%%%%%%%%%%%%%%%%%%%%%%%%%%%%%%%%%%%%%%%%%%%%%%%%%%%%%%%%%%%%%%%%%%%%%%
%%% FOOTNOTE
{\bf{Bootstrapping the programmer's apprentice: Simple interactive behavior for signaling and editing:}}\\
%%% FOOTNOTE
%%% %%%%%%%%%%%%%%%%%%%%%%%%%%%%%%%%%%%%%%%%%%%%%%%%%%%%%%%%%%%%%%%%%%%%%%%%%%%%

In the first stage of bootstrapping, the assistant's automated tutor engages in an analog of the sort of simple signaling and reinforcement that a mother might engage in with her baby in order to encourage the infant to begin taking notice of its environment and participating in the simplest forms of communication. The basic exchange goes something like: the mother draws the baby's attention to something and the baby acknowledges by making some sound or movement. This early step requires that the baby can direct its gaze and attend to changes in its visual field.

In the case of the assistant, the relevant changes would correspond to changes in FIDE or the shared browser window, pointing would be accomplished by altering the contents of FIDE buffers or modifying HTML. Since the assistant has an innate capability to parse language into sequences of words, the tutor can preface each lesson with short verbal lesson summary, e.g., "the variable '{\tt{foo}}'", "the underlined variable", "the highlighted assignment statement", "the expression highlighted in blue". The implicit curriculum followed by the tutor would systematically graduate to more complicated language for specifying referents, e.g., "the {\it{body}} of the '{\tt{for}}' loop", "the '{\tt{else}}' {\it{clause}} in the '{\tt{conditional}} statement", "the {\it{scope}} of the variable '{\tt{counter}}'", "the expression on the {\it{right-hand side}} of the first assignment statement".

The goal of the bootstrap tutor is to eventually graduate to simple substitution and repair activities requiring a combination of basic attention, signaling, requesting feedback and simple edits, e.g., "highlight the scope of the variable shown in red", "change the name of the function to be "{\tt{Increment\_Counter}}", "insert a "{\tt{for}}" loop with an iterator over the items in the "{\tt{bucket}}" list", "delete the next two expressions", with the length and complexity of the specification gradually increasing until the apprentice is capable of handling code changes that involve multiple goals and dozens of intermediate steps, e.g., "delete the variable "{\tt{Interrupt\_Flag}}" from the parameter list of the function declaration and eliminate all of the expressions that refer to the variable within the scope of the function definition".

Note the importance of an attentional system that can notice changes in the integrated development environment and shared browser window, the ability to use recency to help resolve ambiguities, and emphasize basic signals that require noticing changes in the IDE and acknowledging that these changes were made as a means of signaling expectations relevant to the ongoing conversation between the programmer and the apprentice. These are certainly subtleties that will have to be introduced gradually into the curricular repertoire as the apprentice gains experience. We are depending on employing a variant of Riedmiller~\etal{} that will enable us to employ the FIDE to gamify the process by evaluating progress at different levels using a combination of general extrinsic reward and policy-specific intrinsic motivations to guide action selection~\cite{RiedmilleretalCoRR-18}.

Randall O'Reilly mentioned in his class presentation the idea that natural language might play an important role in human brains as an intra-cortical lingua franca. Given that one of the primary roles language serves is to serialize thought thereby facilitating serial computation with all of its advantages in terms of logical precision and combinatorial expression, projecting a distributed connectionist representation through some sort of auto encoder bottleneck might gain some advantage in combining aspects of symbolic and connectionist architectures. This also relates to O'Reilly's discussion of the hippocampal system and in particular the processing performed by the dentate gyrus and hippocampal areas CA1 in CA2 in generating a sparse representation that enables rapid binding of arbitrary informational states and facilitates encoding and retrieving of episodic memory in the entorhinal cortex.\\

%%% %%%%%%%%%%%%%%%%%%%%%%%%%%%%%%%%%%%%%%%%%%%%%%%%%%%%%%%%%%%%%%%%%%%%%%%%%%%%