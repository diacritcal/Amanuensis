%%% %%%%%%%%%%%%%%%%%%%%%%%%%%%%%%%%%%%%%%%%%%%%%%%%%%%%%%%%%%%%%%%%%%%%%%%%%%%% 
%%% %%%%%%%%%%%%%%%%%%%%%%%%%%%%%%%%%%%%%%%%%%%%%%%%%%%%%%%%%%%%%%%%%%%%%%%%%%%% 

This document is not intended to provide the reader with a short course in cognitive science, artificial intelligence, natural language processing, machine learning, artificial neural networks, or automated code synthesis / automatic inductive programming, and is certainly not intended to cover all these disciplines in any but the most cursory of detail. The primary goal is to explore the possibility of building digital assistants that considerably extend our ability to solve complex engineering problems with a emphasis here on software engineering. A secondary goal is to explain how the field of neuroscience might help to achieve our primary goal. 

The fields of cognitive and systems neuroscience are playing an important role in directing and accelerating research on artificial neural network systems. Much of this work predates and helped give rise to the especially exciting work on connectionist models in the 1980s. However, in the nearly 40 intervening years, a great deal of progress has been made much of it due to improved methods for studying the behavior of awake behaving animal subjects and human beings in particular. Indeed, this work is undergoing a renaissance fueled by even more powerful methods for observing brain activity in human beings in the midst of solving complex cognitive tasks.

In contrast, the field the automatic programing / code synthesis, after decades of steady, often quite practical but not particularly remarkable progress on using symbolic methods \emdash{} much of it originating in labs outside the United States, is seeing a resurgence of research instigated in large part by the renewed interest and substantial progress on artificial neural networks. It remains to be seen whether artificial neural networks will have a significant impact on code synthesis, however there appear to be opportunities to leverage what we know about both natural and artificial neural networks to make progress, and hybrid systems that combine both connectionist and traditional symbolic methods may have the best chance of pushing the state-of-the-art significantly beyond its present level.

%%% %%%%%%%%%%%%%%%%%%%%%%%%%%%%%%%%%%%%%%%%%%%%%%%%%%%%%%%%%%%%%%%%%%%%%%%%%%%%
%%% %%%%%%%%%%%%%%%%%%%%%%%%%%%%%%%%%%%%%%%%%%%%%%%%%%%%%%%%%%%%%%%%%%%%%%%%%%%%

We begin with the problem of how we represent information. In the case of the programmer's apprentice relevant sources of information include the obvious things that software engineers think about algorithms, programs, subroutines, interfaces, tools such as compilers, interpreters, parsers, debuggers, etc. Then there are the things that programmer's generally don't think about explicitly but that involve how they solve problems and organize their thoughts, including design strategies we learn in computer science courses like divide and conquer, recursion and dynamic programming. Finally, there is information of a sort that plays a role in any individual or collaborative effort such as short- and long-term memory including episodic memory, how plans, tasks, concrete facts, and abstract thoughts are encoded in such memories.

%%% %%%%%%%%%%%%%%%%%%%%%%%%%%%%%%%%%%%%%%%%%%%%%%%%%%%%%%%%%%%%%%%%%%%%%%%%%%%%
%%% %%%%%%%%%%%%%%%%%%%%%%%%%%%%%%%%%%%%%%%%%%%%%%%%%%%%%%%%%%%%%%%%%%%%%%%%%%%%

%%% PART I

%%% \rawhtml
%%% <a name="integrated_architecture_part_one"></a>
%%% \endrawhtml
%%% \subsection*{May 15, 2018}

\subsubsection*{PART I}

Rushed as usual, but had some thoughts about what an integrated architecture might look like that I wanted to write down and share. I'm not alone in thinking that one thing missing from the ideas that we've been hearing about has to do with the hierarchical structure of language, planning, and, more generally, thought. A couple of teams seem to be thinking along these lines and I've suggested some formal models as possible approaches to generating data with such hierarchical, recursive structure. Hierarchical hidden Markov models (HHMM) were introduced by Fine, Singer and Tishby~\cite{FineetalML-98} to account for a wide range of natural processes and I employed them in developing models of the neocortex in 2005~\cite{DeanAAAI-05,DeanAIMATH-06,DeanAMAI-06}

While at Numenta in 2006, I experimented with methods for segmenting sequences generated by mixtures of switching variable memory Markov sources~\cite{SeldinetalICML-01,ApostolicoandBejeranoCB-00,SeldinMSc2001}. The general problem of learning such models is computational intractable, and the learning methods of the day generally fell short when applied to problems of interest. In the following exercise, rather than immediately focus on learning such models, we start by considering the simpler problem of how we might represent {\it{subroutines}} that implement simple program transformations in a neural-network architecture.

We begin with the simplifying assumption that subroutines can be represented as tuples consisting of a set of operands represented as high-dimensional embedding vectors, a weight matrix representing the transformation and a product vector space in which to embed the result. In applying this idea to program transformations, assume that each operand corresponds to the embedding of an abstract-syntax-tree representation of a code fragment, w.l.o.g., any non-terminal node in the AST of a syntactically well-formed program. 
\rawhtml
<a name="convention_and_abbreviation_index"></a>
\endrawhtml
In the remainder of this entry and the {\urlh{#integrated_architecture_part_two}{next}}, we use the following abstractions and abbreviations:
%
\begin{itemize}
%
\item {\it{prefrontal cortex}} (PFC) including attention, conscious access, reward-based-learning and executive control~\cite{WangetalNATURE-NEUROSCIENCE-18,KrieteetalPNAS-13};
%
\item {\it{entorhinal-hippocampal complex}} (EHC) in its role as primary interface between the hippocampus and neocortex~\cite{OReillyetalCS-15,OReillySCIENCE-06};
%
\item {\it{global workspace}} (GW) broadly distributed cortical circuits connected through long-range excitatory axons\footnote{%
%
    Dehaene~\etal{DehaeneetalPNAS-98} distinguish two main computational spaces within the brain: "The first is a {\it{processing network}}, composed of a set of parallel, distributed and functionally specialized processors or modular sub-systems ranging from primary sensory processors (such as area V1) or unimodal processors (such as area V4), which combine multiple inputs within a given sensory modality, up to heteromodal processors (such as the visuo-tactile neurons in area LIP) that extract highly processed categorical or semantic information. Each processor is subsumed by topologically distinct cortical domains with highly-specific local or medium-range connections that {\it{encapsulate}} information relevant to its function. The second computational space is a {\it{global workspace}}, consisting of a distributed set of cortical neurons characterized by their ability to receive from and send back to homologous neurons in other cortical areas horizontal projections through long-range excitatory axons (which may impinge on either excitatory or inhibitory neurons). Our view is that this population of neurons does not belong to a distinct set of {\it{cardinal}} brain areas but, rather, is distributed among brain areas in variable proportions."};~\cite{DehaeneetalPNAS-98};
%
\item {\it{basal ganglia}} (BG) for its role in action selection and dynamic gating input to the prefrontal cortex\cite{OReillyetalLEABRA-16,KrieteetalPNAS-13};
% 
\item {\it{semantic memory system}} (SMS) including areas of the brain responsible for mathematical and abstract thought~\cite{Tulving1972,BinderandDesaiTiCS-11};
%
\item {\it{episodic memory system}} (EMS) including episodic memory management and memory-based parameter adaptation~\cite{SprechmannetalICLR-18,PritzeletalICML-17};
%
\item {\it{differentiable neural computer}} (DNC) as the interface to the integrated development environment prostheses~\cite{GravesetalNATURE-16,GravesetalCoRR-14};
%
\item {\it{abstract syntax-tree}} (AST) is a representation of the abstract syntactic structure of a source-code program\footnote{%
%
  Here is the abstract syntax tree for {\urlh{https://en.wikipedia.org/wiki/Euclidean_algorithm}{Euclid's algorithm}} which is an efficient method for computing the greatest common divisor (GCD) of two numbers:
  % 
  \begin{center}
    \includegraphics[width=6.0in]{./figures/Euclids_Greatest_Common_Divisor_Method.png}
  \end{center}}~\cite{DevlinetalICLR-18,WangetalCoRR-17};
% 
\end{itemize}

%%% %%%%%%%%%%%%%%%%%%%%%%%%%%%%%%%%%%%%%%%%%%%%%%%%%%%%%%%%%%%%%%%%%%%%%%%%%%%%
%%% %%%%%%%%%%%%%%%%%%%%%%%%%%%%%%%%%%%%%%%%%%%%%%%%%%%%%%%%%%%%%%%%%%%%%%%%%%%%

%%% PART II

%%% \rawhtml
%%% <a name="integrated_architecture_part_two"></a>
%%% \endrawhtml
%%% \subsection*{May 17, 2018}

%%% \begin{verbatim}
%%% %%% Thu May 17 05:41:08 PDT 2018
%%% \end{verbatim}

\subsubsection*{PART II}

Referring to the abstractions and abbreviations introduced in the previous {\urlh{#integrated_architecture_part_one}{entry}}, reading a program from {\tt{STDIO}} \emdash{} the analog of a human programmer reading a program displayed on a monitor \emdash{} will result in \emdash{} at least \emdash{} two different internal representations of the resulting AST: an embedding vector in the SMS and a key-value representation in the DNC. The former allows us manipulate programs and program fragments as fully-differentiable representations within distributed models. The latter allows us to modify, execute and share code in a human-accessible format, fully compatible with our software-development toolchain.

Following~\cite{PritzeletalICML-17}, we assume EMS consists of initial-state-action-reward-next-state tuples of the form $(s_{t},\;a_{t},\;r_{t},\;s_{t+1})$. State representations $s_{t}$ have to be detailed enough to reconstruct the context in which the action is performed and yet concise enough to be practical. Suppose the PFC directs the activation of selected circuits in the SMS via the GWS in accord with Dehaene~\etal{}~\cite{DehaeneetalSCIENCE-17,Dehaene2014} assuming a prior that generates low-dimensional thought vectors~\cite{BengioCoRR-17}. $s_{t}$ encodes the attentional state that served to identify representations in SMS relevant to $a_{t}$ allowing the EHC to produce the resulting state $s_{t+1}$. Given $s_{t}$ we can reproduce the activity recorded in the EMS, and, in principle, incorporate multiple steps and contingencies in a policy constituting a specialized program-synthesis or program-repair subroutine.

Such subroutines would include repairing a program in which a variable is introduced but not initialized, or when it is initialized but ambiguously typed or scoped. As another example, a variable is initialized as {\tt{VOID}} and subsequently assigned an integer value in some but not all branches of a conditional statement. Other examples of repair routines include problems with the use of comparison operators, e.g., conditional branches both with \hmleq{}, the {\tt{is}} operator is used instead of {\tt{is not}}, or vice versa, confusion involving {\tt{A is not None}}, {\tt{A not None}} and {\tt{A != None}}, and problems involving class methods, e.g., when {\tt{self}} accessor is missing from a variable, e.g., {\tt{mode = 'manual'}} instead of {\tt{self.mode = 'manual'}}~\cite{ShinetalICLR-18b,DevlinetalICLR-18,WangetalCoRR-17}.

Attentional machinery in the prefrontal cortex (PFC) populates the global workspace (GWS) by activating circuits relevant to the current input and internal state, including that of the DNC and any ongoing activity in (SMS) circuits produced by previous top-down attention and bottom-up sensory processing. The PFC in its role as executive arbiter identifies operators in the form of policy subroutines and then enlists the EHC to \emdash{} using terminology adapted from Von Neumann machines \emdash{} to load registers in short-term memory and perform operations by using fast weights to transform the contents of the loaded registers into product representations that can either be fed to associative embeddings, temporarily stored in other registers or used to modify the contents of the DNC thereby altering the AST representation of the target code and updating the display to provide feedback to the human programmer.

What's left out of this account so far includes how we might take advantage of semantics in the form of executing code and examining traces in order to better understand the consequences of the changes just made. Presumably, wrapping a code fragment in a function and executing the function with different input to examine changes in the state variables could be used as a distal reinforcement signal providing intermediate rewards useful in debugging subroutines. As pointed out earlier, subroutines designed to modify code are likely to involve many conditional choices and so it is important for subroutine policies to be highly conditioned on the status of specific state variables. Indeed a technique such as model-based parameter adaptation may be perfectly suited to providing such context-sensitive adaptations.

Perhaps this next thought seems obvious, but it is worth keeping in mind that the human brain does a great deal of (parallel) processing that never rises to the level of conscious attention. The executive control systems in the prefrontal cortex don't have micromanage everything. Every thought corresponds to a pattern of activity in one or more neural circuits in the brain or beyond in the peripheral nervous system. One pattern of activity inevitably leads to another in the same or another set of neurons. For example, patterns of activity that begin in the sensory cortex can lead to patterns of activity in the motor cortex and can have consequences elsewhere in the brain, e.g., in the cerebellar cortex resulting in speech, or external to the central nervous system as in the case of neurons that propagate through the peripheral nervous system causing muscles to contract and extend thereby making your limbs and torso move. 

Every new observation, every act of creating a new thought or revisiting an old one produces even more activity in the brain resulting in new thoughts some of which are ignored as their reverberations weaken and die and others that spawn new thoughts and proliferate under the influence of reentrant production of activity and the active encouragement of conscious attention in a perpetually self reinforcing, reimagining and self-analyzing cycle of recurrent activity. Meta-reinforcement learning supports the sort of diverse activity one might expect from a system that selects activity to attend to and then makes available in the global workspace for ready access by other systems. Sustaining a collection of such activated circuits would help to provide a context, serve to maintain a stack of policies, guide switching between them, support caching partial results for later use, reconstructing necessary state as needed when restoring policy after a recursive descent.

When you think of building systems that can develop new algorithms it is instructive the think about the simple case of learning to sort lists from input-output pairs. I would guess that bubble sort would be the easiest to come up with, but even then it is easier if you start with simple I/O pairs like $[A,\;B] \hmrarr{} [A,\;B], [B,\;A] \hmrarr{} [A,\;B]$ and work up to longer lists. As Dan Abolafia pointed out in his class {\urlh{https://web.stanford.edu/class/cs379c/calendar_invited_talks/lectures/04/24/index.html}{presentation}}, it is relative easy to learn to sort lists of length no more than $n$, but substantially more difficult to learn an algorithm that works for lists of arbitrary length, without the ability to construct a simple inductive proof of correctness. Logic and basic theorem proving are certainly important in learning to write programs. You might want to look at the Coq proof assistant\footnote{%
%
  {\urlh{https://en.wikipedia.org/wiki/Coq}{Coq}} is a formal proof management system that "provides a formal language to write mathematical definitions, executable algorithms and theorems together with an environment for semi-interactive development of machine-checked proofs" \emdash{} excerpted from Mike Nahas' {\urlh{https://coq.inria.fr/tutorial-nahas}{tutorial}}. The Coq formal language is also a programming language. It was developed by and named after its inventor, Thierry Coquand, and employed by Georges Gonthier and Benjamin Werner to create a new proof of the {\urlh{https://en.wikipedia.org/wiki/Four_color_theorem}{Four Color Theorem}} first proved by Kenneth Appel and Wolfgang Haken using a combination of human and computer theorem proving techniques.} for a glimpse at the future of algorithm development.

%%% %%%%%%%%%%%%%%%%%%%%%%%%%%%%%%%%%%%%%%%%%%%%%%%%%%%%%%%%%%%%%%%%%%%%%%%%%%%%
%%% %%%%%%%%%%%%%%%%%%%%%%%%%%%%%%%%%%%%%%%%%%%%%%%%%%%%%%%%%%%%%%%%%%%%%%%%%%%%

%%% PART III

%%% \rawhtml
%%% <a name="integrated_architecture_part_three"></a>
%%% \endrawhtml
%%% \subsection*{May 19, 2018}

%%% \begin{verbatim}
%%% %%% Sat May 19 03:45:38 PDT 2018
%%% \end{verbatim}

\subsubsection*{PART III}

My objective this morning is to finish working on Figure~{\urlh{#fig_Integrated_Architecture_Integrated_Figure}{48}} describing a possible integrated architecture and then try to explain the architecture in the simplest way possible. By that I mean explaining the {\it{function}} of each component without going into detail about its {\it{implementation}}. Yesterday evening I made an inventory of functions associated with the proposed architecture along with tools that might be used to implement them, e.g., active maintenance of circuits in the global workspace (WS) \hmrarr{} fast weights / dynamic linking~\cite{SchlagandSchmidhuberICLR-18,BaetalCoRR-16} and dynamic gating of bistable activity in the prefrontal cortex (PFC) \hmrarr{} gated recurrent networks~\cite{AnonymousICLR-18b,ChoetalCoRR-15}. Most functions can probably be implemented using variations on standard models or by leveraging ideas from cognitive neuroscience that we have been exploring in class. In any case, a full accounting will have to wait. 

%%% %%%%%%%%%%%%%%%%%%%%%%%%%%%%%%%%%%%%%%%%%%%%%%%%%%%%%%%%%%%%%%%%%%%%%%%%%%%% 

\setcounter{figure}{47}

%%% %%%%%%%%%%%%%%%%%%%%%%%%%%%%%%%%%%%%%%%%%%%%%%%%%%%%%%%%%%%%%%%%%%%%%%%%%%%% 

%%% Figure~{\urlh{#fig_Integrated_Architecture_Integrated_Figure}{48}}
\rawhtml
<a name="fig_Integrated_Architecture_Integrated_Figure"></a>
\endrawhtml
\begin{figure}
%
  \hrule{}
%
  \begin{center}
%%% \includegraphics[width=11.0in]{./figures/Integrated_Architecture_Integrated_Figure.png}
    \includegraphics[width=11.0in]{./figures/Integrated_Architecture_Integrated_Figure.jpg}
  \end{center}
%
  \caption{The triptych on the bottom concatenates the graphics from three papers by Randal O'Reilly and his colleagues. They are reproduced here along with their original captions for your convenience in understanding the graphic in the top panel labeled {\rawhtml<span style="color:red">(a)</span>\endrawhtml} that is explained in detail in the main text. Panel {\rawhtml<span style="color:red">(a)</span>\endrawhtml} follows the shape and color conventions employed in Panel {\rawhtml<span style="color:red">(c)</span>\endrawhtml} except the yellow square shapes that denote abstract structures and not anatomical features. The acronyms are expanded and explained {\urlh{#convention_and_abbreviation_index}{here}}.
    %%% OReillySCIENCE-06_Figure_02_Dynamic.png
    Figure~2 in O'Reilly~\cite{OReillySCIENCE-06} shown in Panel {\rawhtml<span style="color:red">(b)</span>\endrawhtml} \emdash{} Dynamic gating produced by disinhibitory circuits through the basal ganglia and frontal cortex/PFC (one of multiple parallel circuits shown).  In the base state (no striatum activity) and when {\tt{NoGo}} (indirect pathway) striatum neurons are firing more than {\tt{Go}}, the SNr (substantia nigra pars reticulata) is tonically active and inhibits excitatory loops through the basal ganglia and PFC through the thalamus. This corresponds to the gate being closed, and PFC continues to robustly maintain ongoing activity (which does not match the activity pattern in the posterior cortex, as indicated). When direct pathway {\tt{Go}} neurons in striatum fire, they inhibit the SNr and thus disinhibit the excitatory loops through the thalamus and the frontal cortex, producing a gating-like modulation that triggers the update of working memory representations in prefrontal cortex. This corresponds to the gate being open.
    %%% OReillyetalLEABRA-16_Figure_07_Leabra.png
    Figure~7 in O'Reilly~\etal{}~\cite{OReillyetalLEABRA-16} shown in Panel {\rawhtml<span style="color:red">(c)</span>\endrawhtml} \emdash{} The macrostructure of the {\urlh{https://en.wikipedia.org/wiki/Leabra}{Leabra}} architecture, with specialized brain areas interacting to produce overall cognitive function.
    %%% OReillyetalLEABRA-16_Figure_08_Leabra.png
    Figure~8 in O'Reilly~\etal{}~\cite{OReillyetalLEABRA-16} shown in Panel {\rawhtml<span style="color:red">(d)</span>\endrawhtml} \emdash{} Structure of the hippocampal memory system and associated medial temporal lobe cortical structures including entorhinal cortex.}
%
  \hrule{}
%
\end{figure}

%%% %%%%%%%%%%%%%%%%%%%%%%%%%%%%%%%%%%%%%%%%%%%%%%%%%%%%%%%%%%%%%%%%%%%%%%%%%%%% 

Figure~{\urlh{#fig_Integrated_Architecture_Integrated_Figure}{48}} shows a diagram of the human brain overlaid with a simplified architectural drawing. The box shapes represent abstract systems and the oval and triangular shapes represent anatomical features for which we can supply computational models. For example, the box labeled GW represents the global workspace which performs particular function in the architecture, but actually spans a good portion of the neocortex. Whereas the triangle labeled BG represents a group of subcortical nuclei called the basal ganglia situated at the base of the forebrain.

The box labeled AST represents a form of sensory input corresponding to the ingestion of abstract syntax trees representing code fragments. The oval labeled SMS represents semantic memory and the box labeled DNC corresponds to a differentiable neural computer. When the system ingests a new program fragment the resulting AST is encoded in the SMS as an embedding vector and simultaneously as a set of key-value pairs in the DNC. Here we think of the DNC as a body part or external prosthesis with corresponding maps in the somatosensory and motor cortex that enable reading and writing respectively.

Our explanation of the architecture proceeds top down, as it were, with a discussion of executive function in the prefrontal cortex. The GWS provides two-way connection between structures in the prefrontal cortex and homologous structures of a roughly semantic character throughout the rest of neocortex thereby enabling the PFC to listen in on diverse circuits in the neocortex and select a subset of such circuits for attention. Stanislas Dehaene describes this process as the function of consciousness, but we need not commit ourselves to such interpretation here.

Not only does the PFC selectively activate circuits but it can also maintain the activity such circuits indefinitely as constituents of working memory. Since this capability is limited by the capacity of the PFC, the content of working memory is limited and adding new constituents may curtail the activation of existing constituents. In practice, we intend to model this capability using meta-reinforcement learning~\cite{WangetalNATURE-NEUROSCIENCE-18} (MRL) in which the MRL system relies on the GWS network to sample, evaluate and select constitute circuits guided by a suitable prior~\cite{BengioCoRR-17} and past experience and then maintain their activity by a combination of memory networks~\cite{WestonetalCoRR-14} and fast weights~\cite{BaetalCoRR-16}. 

%%% %%%%%%%%%%%%%%%%%%%%%%%%%%%%%%%%%%%%%%%%%%%%%%%%%%%%%%%%%%%%%%%%%%%%%%%%%%%% 

Meta-reinforcement learning serves a second complementary role in the PFC related to executive function. We will refer to the first role as MRL-A for "attention" and the second as MRL-P for "planning". MRL-A is trained to focus attention on relevant new sensory input and new interpretations of and associations among prior perceptions and thoughts. MRL-P is trained to capitalize on and respond to opportunities made available by new and existing constituents in working memory. Essentially MRL-P is responsible for the scheduling and deployment of plans relevant to recognized opportunities to act. These plans are realized as policies trained by reinforcement learning from traces of past experience or constructed on the fly in response to unexpected / unfamiliar contingencies by recovering and reimagining past activities recovered from episodic memory.

MRL-A and MRL-P could be implemented as a single policy, but it is simpler to think of them as two coupled systems, one responsible for focusing attention by constantly assessing changes in (neural) activity throughout the global workspace, and a second responsible for overseeing the execution of plans in responding to new opportunities to solve problems. MRL-A is as a relatively straightforward reinforcement learning system independently performing its task largely a function of whatever neural activity is going on in the GW, its attentional network and the prior baked into its reward function. MRL-P could be implemented along the lines of the Imagination-Augmented Agent (I2A) architecture~\cite{WeberetalCoRR-17} or the related Imagination-Based Optimization~\cite{HamricketalCoRR-17} and Imagination-Based Planning~\cite{PascanuetalCoRR-17} systems.

%%% %%%%%%%%%%%%%%%%%%%%%%%%%%%%%%%%%%%%%%%%%%%%%%%%%%%%%%%%%%%%%%%%%%%%%%%%%%%%

%%% \begin{verbatim}
%%% %%% Mon May 21 04:31:07 PDT 2018
%%% \end{verbatim}

%%% %%%%%%%%%%%%%%%%%%%%%%%%%%%%%%%%%%%%%%%%%%%%%%%%%%%%%%%%%%%%%%%%%%%%%%%%%%%%

The remaining parts of the architecture involve the interplay between the PFC and the semantic and episodic memory systems as facilitated by the basal ganglia and hippocampus. If we had a policy pre-trained for every possible contingency, we would be nearly done \emdash{} let MRL-A draw attention to relevant internal and external activity and then design a simple just-in-time greedy scheduler that picks the policy with the highest reward given the state vector corresponding to the current content of working memory. Unfortunately, the life of an apprentice programmer is not nearly so simple.

The apprentice might listen to advice from a human programmer or watch someone solve a novel coding problem or repair a buggy program. Alternatively, it may be relatively simple to adapt an existing policy to work in the present circumstances. However, making progress on harder problems will depend on expert feedback or having an existing reward function that generalizes to the problem at hand. In the remainder of this entry, we set aside these problems for another day and concentrate on the basic functionality provided by the basal ganglia as highlighted in Panel {\rawhtml<span style="color:red">(c)</span>\endrawhtml} \emdash{} of Figure~{\urlh{#fig_Integrated_Architecture_Integrated_Figure}{48}}.

The basal ganglia in cognitive models such as the one described by Randall O'Reilly's in his {\urlh{https://web.stanford.edu/class/cs379c/calendar_invited_talks/lectures/04/12/index.html}{presentation}} in class, play a central role in action selection. This seems like a good opportunity to review how actions are represented in deep-neural-network implementations of reinforcement learning. Returning to our default representation for the simplest sort of episodic memory, $(s_{t},\;a_{t},\;r_{t},\;s_{t+1})$, it’s easy to think of a state $s$ as a vector $s \hmisin{} \hmreals{}^{n}$ and a reward $r$ as a scalar value, $r \hmisin{} \hmreals{}$, but how are actions represented?

Most approaches to deep reinforcement learning employ a tablular model of the policy implying a finite \emdash{} and generally rather small \emdash{} repertoire of actions. For example, most of the experiments described in Wayne~\etal{}~\cite{WayneetalCoRR-18} (MERLIN) six-dimensional one-hot binary vector that maps a set of six actions: move forward, move backward, rotate left with rotation rate of 30, rotate right with rotation rate of 30, move forward and turn left, move forward and turn right. The action space for the grid-world problems described in Rabinowitz~\etal{}~\cite{RabinowitzetalCoRR-18} (ToMnets) consists of four movement actions: up, down, left, right and stay.

The remainder of this discussion will have to wait until I have a little more time. It will be posted {\urlh{#integrated_architecture_part_four}{here}}.

%%% %%%%%%%%%%%%%%%%%%%%%%%%%%%%%%%%%%%%%%%%%%%%%%%%%%%%%%%%%%%%%%%%%%%%%%%%%%%%
%%% %%%%%%%%%%%%%%%%%%%%%%%%%%%%%%%%%%%%%%%%%%%%%%%%%%%%%%%%%%%%%%%%%%%%%%%%%%%%

%%% PART IV

%%% \rawhtml
%%% <a name="integrated_architecture_part_four"></a>
%%% \endrawhtml
%%% \subsection*{May 21, 2018}

%%% \begin{verbatim}
%%% %%% Mon May 21 05:52:18 PDT 2018
%%% \end{verbatim}

\subsubsection*{PART IV}

Continuing from the previous log {\urlh{#integrated_architecture_part_three}{entry}}, the programmer's apprentice (PA) operates on programs represented as trees, where the set of actions includes basic operations for traversing and editing trees \emdash{} or more generally directed-graphs with cycles if you assume edges in abstract syntax trees corresponding to loops, recursion and nested procedure calls, i.e., features common to nearly all the programs we actually care about. We still have a finite number of actions since for any given project we can represent the code base as a directed-acylic graph with annotations to accommodate procedure calls and recursion, and use attention to direct and contextualize a finite set of edit operations\footnote{%
%
  While the apprentice operates directly on the AST representation of the code, the IDE can be designed to periodically coerce this representation into a syntactically-correct form, display the result as human-readable code, and display meaningful annotations that highlight program fragments relevant to the ongoing collaboration and track the apprentice's attention.}.

Pritzel~\etal{}~\cite{PritzeletalCoRR-17} employ a semi-tabular representation of an agent's experience of the environment possessing features of episodic memory including long-term memory, sequentiality and context-based lookups. The representation called a {\it{differential neural dictionary}} (DND) is related to Graves~\etal{}~\cite{GravesetalNATURE-16} DNC. The programmer's apprentice is better suited to Vinyals~\etal{}~\cite{VinyalsetalNIPS-15} related idea of a {\it{pointer-network}} designed to learn the conditional probability of an output sequence with elements that are discrete tokens corresponding to positions in an input sequence \emdash{} see related work in natural language processing by Merity~\etal{}~\cite{MerityetalCoRR-16} on {\it{pointer sentinels}}.

%%% %%%%%%%%%%%%%%%%%%%%%%%%%%%%%%%%%%%%%%%%%%%%%%%%%%%%%%%%%%%%%%%%%%%%%%%%%%%% 

\setcounter{figure}{48}

%%% %%%%%%%%%%%%%%%%%%%%%%%%%%%%%%%%%%%%%%%%%%%%%%%%%%%%%%%%%%%%%%%%%%%%%%%%%%%% 

%%% Figure~{\urlh{#fig_SeeetalACL-17_Figure_03}{49}}
\rawhtml
<a name="fig_SeeetalACL-17_Figure_03"></a>
\endrawhtml
\begin{figure}
%
  \hrule{}
%
  \begin{center}
    \includegraphics[width=11.0in]{./figures/SeeetalACL-17_Figure_03.png}
  \end{center}
%
  \caption{Pointer-generator model. For each decoder timestep a generation probability $P\mbox{\rm{gen}} \hmisin{} [0, 1]$ is calculated, which weights the probability of {\it{generating}} words from the vocabulary, versus {\it{copying}} words from the source text. The vocabulary distribution and the attention distribution are weighted and summed to obtain the final distribution, from which we make our prediction. Note that out-of-vocabulary article words such as 2-0 are included in the final distribution. \emdash{} adapted from~\cite{SeeetalACL-17}.}
%
  \hrule{}
%
\end{figure}

%%% %%%%%%%%%%%%%%%%%%%%%%%%%%%%%%%%%%%%%%%%%%%%%%%%%%%%%%%%%%%%%%%%%%%%%%%%%%%% 

\rawhtml
<a name="automated_programming_code_repair"></a>
\endrawhtml

One approach involves representing a program as an abstract syntax tree and performing a series of repairs that involve replacing complete subtrees in the AST. It might be feasible to use some variant of the pointer-network concept, e.g., {\cite{BhoopchandetalICLR-17}}, {\cite{SeeetalACL-17}} and  {\cite{WangandJiangICLR-17}} or neural programmer framework {\cite{NeelakantanetalICLR-17}}, but there are limitations with all of the alternatives I've run across so far, requiring additional innovation to deal with the dynamic character of editing AST representations, but at least the parsing problem is solved for us \emdash{} all we have to do is make sure that our edits maintain syntactic well-formedness.

Most of the existing pointer-network applications analyze / operate on a fixed structure such as a road map, e.g., examples include the planar graphs that Oriol Vinyals addresses in his paper~\cite{VinyalsetalNIPS-15}, recognizing long-range dependencies in code repositories~\cite{BhoopchandetalICLR-17}, and annotating text to support summarization~\cite{SeeetalACL-17}. Student projects focusing on program-repair might try ingesting programs using an LSTM, creating a pointer-network / DNC-like representation of the AST and then altering selected programs by using fragments from other programs, but be advised this approach will likely require inventing extensions to existing pointer-network techniques.

One possibility for training data is to use the ETH / SRI Python {\urlh{https://www.sri.inf.ethz.ch/py150}{dataset}} that was developed by Veselin Raychev as part of his {\urlh{https://www.sri.inf.ethz.ch/raychev_thesis.pdf}{thesis}} on automated code synthesis\footnote{%
%
  From the ETH/SRI website: "We provide a dataset consisting of parsed Parsed ASTs that were used to train and evaluate the DeepSyn tool. The Python programs are collected from GitHub repositories by removing duplicate files, removing project forks (copy of another existing repository), keeping only programs that parse and have at most 30K nodes in the AST and we aim to remove obfuscated files. Furthermore, we only used repositories with permissive and non-viral licenses such as MIT, BSD and Apache. For parsing, we used the Python AST parser included in Python 2.7. We also include the parser as part of our dataset. The dataset is split into two parts \emdash{} 100K files used for training and 50K files used for evaluation."}.
%
Possible projects include designing a rewrite system for code synthesis based on NLP work from Chris Manning's lab led by Abigail See~\cite{SeeetalACL-17} focusing on text summarization leveraging pointer networks \emdash{} see Figure~{\urlh{#fig_SeeetalACL-17_Figure_03}{49}} for an excellent schematic overview of their method. Further afield are program synthesis papers that work starting from specifications like Shin~\etal{}~\cite{ShinetalICLR-18b} out of Dawn Song's lab or recent work from Rishabh Singh and his colleagues~\cite{WangetalCoRR-17}.

Another possibility is to use RL to learn repair rules that operate directly on the AST using various strategies. It's not necessary in this case to represent the AST as a pointer network, but, rather, take the expedient of simply creating a new embedding edited AST after each repair. We can generate synthetic data by taking correct programs from the ETH/SRI dataset and introducing bugs and then use these to generate a reward signal, with harder problems requiring two or three separate repairs. 

It might also be worth exploring the idea of working with program embedding vectors in a manner similar to performing arithmetic operations on word vectors in order to recover analogies \emdash{} see the analysis of Levy and Goldberg~\cite{LevyandGoldbergCONIL-14} in which they demonstrate that analogy recovery is not restricted to simple neural word embeddings. For example, given the AST for a program {\tt{P}} with subtree {\tt{Q}} and two possible repairs that correspond to replacing {\tt{Q}} with either {\tt{R}} or {\tt{R'}}, can we determine which is the better outcome {\tt{A = P - Q + R}} or {\tt{A' = P - Q + R'}} and might it serve as a distal reward signal to expedite training?

%%% \begin{verbatim}
%%% %%% Wed May 23 9:24:25 PDT 2018
%%% \end{verbatim}

I also recommend Reed and de Freitas~\cite{ReedandDeFreitasCoRR-15} for its application of the idea of using dynamically programmable networks in which the activations of one network become the weights (program) of another network.  The authors note that this approach was mentioned in Sigma-Pi units section of Rumelhart~\etal{}~\cite{RumelhartetalPDP-86b}, appeared in Sutskever and Hinton~\cite{SutskeverandHintonNIPS-09} in the context of learning higher order symbolic relations and in Donnarumma~\etal{}~\cite{DonnarummaetalIJNS-15} as the key ingredient of an architecture for prefrontal cognitive control.

%%% %%%%%%%%%%%%%%%%%%%%%%%%%%%%%%%%%%%%%%%%%%%%%%%%%%%%%%%%%%%%%%%%%%%%%%%%%%%%
%%% %%%%%%%%%%%%%%%%%%%%%%%%%%%%%%%%%%%%%%%%%%%%%%%%%%%%%%%%%%%%%%%%%%%%%%%%%%%%


